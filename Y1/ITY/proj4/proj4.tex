\documentclass[a4paper,11pt]{article}
\usepackage[utf8]{inputenc}
\usepackage[left=2cm,text={17cm, 24cm},top=3cm]{geometry}


\usepackage{amsfonts}
\usepackage{times}
\usepackage[czech]{babel}

\bibliographystyle{czplain}


\newcommand{\myuv}[1]{\quotedblbase #1\textquotedblleft}

\author{Martin Bažík \\ xbazik00@stud.fit.vutbr.cz}
\title{Typografie a publikování \\ 2. projekt}



\begin{document}
	\begin{titlepage}
		\begin{center}
			\textsc{\Huge Vysoké učení technické v~Brně}\\
			\textsc{\huge Fakulta informačních technologií}
					\vspace{\stretch{0.382}}
			\\{\LARGE Typografie a publikování -- 4. projekt\\
				\Huge Bibliografické citácie }
			\vspace{\stretch{0.618}}
		\end{center}
		{\Large \today  \hfill Martin Bažík}	
	\end{titlepage}
	\newpage	
	
	\section{Písmo}
	Vznik písma bol podmienený potrebou zaznamenať ľudskú reč a sprostredkovať myšlienky a dôležité udalosti 
	\cite{Lenka_Bubanova:Graficky_dizajn_--_multimedialna_prirucka}. Od prvých čínskych znakov, ktoré sa objavili už v~7. tisícročí pred naším letopočtom \cite{Rickon:cin}, vznikli mnohé druhy písma od piktografického až po abecedné. Na našom území sa vyvinulo až na latinku s~diakritickými znamienkami. Mnohí si môžu myslieť, že diakritické znamienka vznikli len prednedávnom, avšak je známe, že prvé diakritické znamienka sa objavili už v~stredoveku \cite{Filip_Blazek:Diakr}.
	
	\section{Fonty}
	Font predstavuje kompletnú sadu znakov abecedy jednej veľkosti a štýlu \cite{wiki:font}. Samotné slovo font má svoj pôvod vo francúzštine, kde slovo fonte znamená zlievarenstvo \cite{Harper:dict}, pretože písmená sa pôvodne odlievali pre tlač. Postupom času vzniklo množstvo typov písma, čím vznikla aj nutnosť klasifikovať písma na základe rôznych metód \cite{blackwell200420th}. Medzi základné znaky kvalitného písma patria kvalitné obrysy, prepracované  metrické, kerningové a zliatkové informácie a úplnosť sady znakov \cite{Martin_Cerny:Znakove_sady_v_typografickych_systemech}. No písma nie sú len o~základných tvaroch a čoskoro sa k~nim začali pridávať aj nové vlastnosti. A~práve v~týchto sa objavuje sila typografie, pretože pri správnej šírke a naklonení môžme pridať textu váhu, s~ktorou zapôsobíme na čitateľa \cite{Steven_Heller:Type_as_agent_of_Power}. Kvôli tomu sa využíva zmiešaná sadzba, teda sadzba zložená z~rôznych stupňov a druhou písma \cite{Rybicka:Latex_pro_zacatecniky}.
	
	
	\section{Histórické spracovanie matematických výrazov}
	Diela obsahujúce matematické výpočty tvoria podstatnú časť diel vytvorených pre vedeckú obec. Pôsob spracovania matematických výrazov je preto podstatnou súčasťou práce s~textom a počas histórie vznikli rôzne prístupy. Jedným z~nich je CaminoReal, ktorého užívateľské rozhranie poskytuje interaktívne, syntaxou riadené upravovanie matematických výrazov \cite{Conf:proc}.
	

\newpage
\bibliography{bibfile}

\end{document}
