\documentclass[a4paper,twocolumn,11pt]{article}
\usepackage[utf8]{inputenc}
\usepackage[left=1.5cm,text={18cm, 25cm},top=2.5cm]{geometry}

\usepackage{amsmath}
\usepackage{amsfonts}
\usepackage{amsthm}
\usepackage{times}
\usepackage[czech]{babel}



\newcommand*{\QEDB}{\hfill\ensuremath{\square}}

\theoremstyle{definition}
\newtheorem{definice}{Definice}[section]
\newtheorem{algoritmus}[definice]{Algoritmus}
\newtheorem{veta}{Věta}

\theoremstyle{remark}
\newtheorem*{dokaz}{Důkaz}


\author{Martin Bažík \\ xbazik00@stud.fit.vutbr.cz}
\title{Typografie a publikování \\ 2. projekt}
\date{}

\begin{document}
	\begin{titlepage}
		\begin{center}
			\textsc{{\Huge Fakulta informačních technologií\\
				Vysoké učení technické v~Brně}}\vspace{\stretch{0.382}}
			\\{\LARGE Typografie a publikování -- 2. projekt\\
				Sazba dokumentů a matematických výrazů}
			\vspace{\stretch{0.618}}
		\end{center}
		{\Large 2016 \hfill Martin Bažík}	
	\end{titlepage}
	\newpage

	\section*{Úvod}
	V~této úloze si vyzkoušíme sazbu titulní strany, matematických vzorců, prostředí a dalších textových struktur obvyklých pro technicky zaměřené texty, například rovnice (\ref{eq:first})  nebo definice \ref{def:first} na straně \pageref{def:first}.
	
	Na titulní straně je využito sázení nadpisu podle optického středu s~využitím zlatého řezu. Tento postup byl probírán na přednášce.
	
	\section{Matematický text}
	Nejprve se podíváme na sázení matematických symbolů a výrazů v~plynulém textu. Pro množinu $V$ označuje $\mbox{card}(V)$ kardinalitu $V$.
	Pro množinu $V$ reprezentuje $V^*$ volný monoid generovaný množinou $V$ s~operací konkatenace.
	Prvek identity ve volném monoidu $V^*$ značíme symbolem $\varepsilon$.
	Nechť $V^+=V^*-\{\varepsilon\}$ Algebraicky je tedy $V^+$ volná pologrupa generovaná množinou $V$ s~operací konkatenace.
	Konečnou neprázdnou množinu $V$ nazvěme \emph{abeceda}.
	Pro $w\in V^*$ označuje $|w|$ délku řetězce $w$. Pro $W\subseteq V$ označuje $\mbox{occur}(w,W)$ počet výskytů symbolů z~$W$ v~řetězci $w$ a $\mbox{sym}(w,i)$ určuje $i$-tý symbol řetězce $w$; například sym$(abcd,3) = c$.
	
	Nyní zkusíme sazbu definic a vět s~využitím balíku \tt amsthm\normalfont.

	\begin{definice}\label{def:first}
	\emph{Bezkontextová gramatika} je čtveřice $G = (V,T,P,S)$, kde $V$ je totální abeceda,
	$T \subseteq V$ je abeceda terminálů, $S \in (V~- T)$ je startující symbol a $P$ je konečná množina \emph{pravidel} tvaru $q\colon A~\to \alpha$, kde $A \in (V~- T)$, $\alpha \in V^* $ a $q$ je návěští tohoto pravidla. Nechť $N = V~- T$ značí abecedu neterminálů.
	Pokud $q\colon A~\to \alpha \in P$, $\gamma,\delta \in V^*$ , $G$ provádí derivační krok z~$\gamma A\delta$ do $\gamma\alpha\delta$ podle pravidla $q\colon A~\to \alpha$, symbolicky píšeme 
	$\gamma A\delta \Rightarrow \gamma\alpha\delta \ [q\colon A~\to \alpha] $ nebo zjednodušeně $\gamma A\delta \Rightarrow \gamma\alpha\delta$. Standardním způsobem definujeme $\Rightarrow^m$, kde $m \geq 0$. Dále definujeme 
	tranzitivní uzávěr $\Rightarrow^+$ a tranzitivně-reflexivní uzávěr $\Rightarrow^*$ .
	\end{definice}
	
	Algoritmus můžeme uvádět podobně jako definice textově, nebo využít pseudokódu vysázeného ve vhodném prostředí (například \texttt{algorithm2e}).
	
	\begin{algoritmus}
	\it Algoritmus pro ověření bezkontextovosti gramatiky. Mějme gramatiku $G = (N, T, P, S)$.
	\begin{enumerate}
	\item Pro každé pravidlo $p \in P$ proveď test, zda $p$ na levé straně obsahuje právě jeden symbol z~$N$ .
	\item Pokud všechna pravidla splňují podmínku z~kroku 1, tak je gramatika $G$ bezkontextová.
	\end{enumerate}
	\end{algoritmus}
	
	\begin{definice}
	\textit{Jazyk} definovaný gramatikou $G$ definujeme jako $L(G) = \{w \in T^*|S \Rightarrow^* w\}$ .
	\end{definice}
	
	\subsection{Podsekce obsahující větu}
	
	\begin{definice}
	Nechť $L$ je libovolný jazyk. $L$ je \textit{bezkontextový jazyk}, když a jen když $L = L(G)$, kde $G$ je libovolná bezkontextová gramatika.
	\end{definice}
	
	\begin{definice}
	Množinu $\mathcal{L}_{CF} = \{L|L$ je bezkontextový jazyk\} nazýváme \textit{třídou bezkontextových jazyků}.
	\end{definice}
	
	\begin{veta}\label{veta:first}	
	\textit{Nechť $L_{abc} = \{a^nb^nc^n|n \geq 0\}$. Platí, že $L_{abc} \notin \mathcal{L}_{CF}$.}
	\end{veta}
	
	\begin{dokaz}	
	Důkaz se provede pomocí Pumping lemma pro bezkontextové jazyky, kdy ukážeme, že není možné, aby platilo, což bude implikovat pravdivost věty \ref{veta:first} .\QEDB
	\end{dokaz}
	
	\section{Rovnice a odkazy}
	
	Složitější matematické formulace sázíme mimo plynulý text. Lze umístit několik výrazů na jeden řádek, ale pak je třeba tyto vhodně oddělit, například příkazem \verb+\quad+. 
	
	\begin{equation*} 
		\sqrt[x^2]{y_0^3}\quad\mathbb{N} = \{0,1,2,\ldots\}\quad x^{y^y} \neq x^{yy}\quad z_{i_j} \neq z_{ij}
	\end{equation*}
	
	V~rovnici (\ref{eq:first}) jsou využity tři typy závorek s~různou explicitně definovanou velikostí.

	\begin{eqnarray}\label{eq:first}
		\bigg\{\Big[\left(a + b \right) * c \Big] ^d + 1 \bigg\} & = & x  \\ \lim\limits_{x\to\infty} \frac{\sin^2x + \cos^2x}{4} & = & y \nonumber
	\end{eqnarray}
	
	V~této větě vidíme, jak vypadá implicitní vysázení limity $\lim_{n \to \infty} f\left(n\right)$ v~normálním odstavci textu. Podobně je to i s~dalšími symboly jako $\sum_1^n$ či $\bigcup_{A\in \beta}$ . V~případě vzorce $\lim\limits_{x\to 0}\frac{\sin x}{x} = 1$ jsme si vynutili méně úspornou sazbu příkazem \verb+\limits+.
	
	\begin{eqnarray}
		\int\limits_{a}^{b} f\left(x\right)\mathrm{d}x & = & -\int_b^a f\left(x\right)\mathrm{d}x \\
		\left(\sqrt[5]{x^4}\right)^\prime = \left(x^{\frac{4}{5}}\right)^\prime & = & \frac{4}{5}x^{-\frac{1}{5}} = \frac{4}{5\sqrt[5]{x}} \\
		\overline{\overline{A\vee B}} & = & \overline{\overline{A} \wedge \overline{B}}
	\end{eqnarray}
	
	\section{Matice}
	
	Pro sázení matic se velmi často používá prostředí \texttt{array} a závorky (\verb+\left+, \verb+\right+). 
	
	\[
	\left( 
	\begin{array}{ccc}
		a + b & b - a  \\
		\widehat{\xi + \omega} & \hat{\pi}  \\
		\vec{a} & \overleftrightarrow{AC}   \\
		0 & \beta  \\
	\end{array} \right)
	\]
	
	\[ 
	\text{A} = 
	\left|\left| 
	\begin{array}{cccc}
	a_{11} & a_{12} & \ldots & a_{1n} \\
	a_{21} & a_{22} & \ldots & a_{2n} \\
	\vdots & \vdots & \ddots & \vdots \\
	a_{m1} & a_{m2} & \ldots & a_{mn} \\
	\end{array} 
	\right|\right|
	\]
	
	\[ 
	\left| 
	\begin{array}{cc}
	t & u\\
	v~& w\\
	\end{array}
	\right|
	= tw - uv
	\]
	
Prostředí \texttt{array} lze úspěšně využít i jinde.

\[
\binom {n} {k}
= 
\begin{cases}
\frac{n!}{k!\left(n-k\right)!} & \mathrm{pro}\ 0\leq k\leq n\\
0              & \mathrm{pro}\ k~< 0\ \mathrm{nebo}  \ k~> n\\
\end{cases}
\]	

	
	\section{Závěrem}
	
	V~případě, že budete potřebovat vyjádřit matematickou konstrukci nebo symbol a nebude se Vám dařit jej nalézt v~samotném \LaTeX u, doporučuji prostudovat možnosti balíku maker \AmS-\LaTeX .
	
	
\end{document}
