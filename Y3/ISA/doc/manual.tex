\documentclass[11pt,a4paper]{article}
\usepackage[slovak]{babel}
\usepackage[utf8]{inputenc}
\usepackage{amsmath,amssymb}
\usepackage{graphicx}
\usepackage{float}

\title{LDAP}
\author{Martin Bažík}

\sloppy

\begin{document}

\begin{titlepage}

\newcommand{\HRule}{\rule{\linewidth}{0.5mm}} % Defines a new command for the horizontal lines, change thickness here

\center % Center everything on the page
 
%----------------------------------------------------------------------------------------
%	HEADING SECTIONS
%----------------------------------------------------------------------------------------

\textsc{\LARGE Vysoké Učení Technické v Brne}\\[1.5cm] % Name of your university/college
\textsc{\Large Fakulta Informačných technológií}\\[0.5cm] % Major heading such as course name
\textsc{\large Sieťové aplikácie a správa sietí}\\[0.5cm] % Minor heading such as course title

%----------------------------------------------------------------------------------------
%	TITLE SECTION
%----------------------------------------------------------------------------------------

\HRule \\[0.4cm]
{ \huge \bfseries LDAP server}\\[0.4cm] % Title of your document
\HRule \\[1.5cm]
 
%----------------------------------------------------------------------------------------
%	AUTHOR SECTION
%----------------------------------------------------------------------------------------

% If you don't want a supervisor, uncomment the two lines below and remove the section above
\Large \emph{Autor:}\\
Martin \textsc{Bažík}\\[3cm] % Your name

%----------------------------------------------------------------------------------------
%	DATE SECTION
%----------------------------------------------------------------------------------------

{\large \today}\\[2cm]

%----------------------------------------------------------------------------------------
%	LOGO SECTION
%----------------------------------------------------------------------------------------

\includegraphics[scale=0.8]{img/fitnew.png}\\[1cm] 
\vfill

\end{titlepage}
\tableofcontents
\newpage

\section{Úvod}
Cieľom tohoto projektu je vytvoriť LDAP server, ktorý dokáže na základe databázy záznamov uložených v súbore CSV formátu spracovať a odpovedať na vyhľadávanie protokolu LDAP. Tento server je konkurentný a implementuje jednoduché naviazanie (simple bind) a vyhľadávanie pomocou Search operácie. Vyhľadávanie implementuje filtre and (0), or (1), not (2), equalityMatch (3), substrings (4) a present(7). Tento dokument obsahuje výťah informácií o fungovaní protokolu LDAP a popis implementácie samotného LDAP serveru. 

\section{LDAP Protokol}
\cite{rfc2251}LDAP je protokol, ktorý poskytuje prístup ku adresáru. Protokol je zacielený na správcovské a prehľadávacie aplikácie, ktoré poskytujú vykonávať operácie nad adresárom. Ide o atomické operácie, ktoré sú vykonávané vo vrstve LDAP správ (message layer). Po ukončení spojenia sú nespracované operácie na správovej vrstve zahodené alebo spracované bez odpovede. Synchronizácia akcií medzi klientom a serverom nie je potrebná.
\linebreak

\noindent\textbf{Kľúčové aspekty LDAP protokolu:}
\begin{itemize}
	\item Popísaný ASN.1 notáciou.
	\item Prenášaný priamo nad TCP alebo iným transportným protokolom
	\item Všetky dátové objekty sú kódované ako bežný reťazec
	\item SASL môže byť použité na zabezpečenie
	\item Hodnoty a názvy musia využívať znakovú sadu ISO 10646
	\item Schéma je zverejnená v adresári pre klienta
\end{itemize}

\subsection{Správa\label{message}}
\cite{rfc4511}Všetky protokolové operácie sú zabalené do obálky LDAPMessage:
\begin{figure}[H]
\begin{verbatim}
        LDAPMessage ::= SEQUENCE {
             messageID       MessageID,
             protocolOp      CHOICE {
                  bindRequest           BindRequest,
                  bindResponse          BindResponse,
                  unbindRequest         UnbindRequest,
                  searchRequest         SearchRequest,
                  searchResEntry        SearchResultEntry,
                  searchResDone         SearchResultDone,
                  ...,
                  intermediateResponse  IntermediateResponse },
             controls       [0] Controls OPTIONAL }

        MessageID ::= INTEGER (0 ..  maxInt)

        maxInt INTEGER ::= 2147483647 -- (2^^31 - 1) --
\end{verbatim}
\caption{Formát LDAP správy}
\end{figure}

\subsubsection{Riešenie chýb}
V prípade, že značka LDAPMessage nemôže byť rozoznaná, messageID zanalyzovaná, protocolOp nie je rozpoznaný ako požiadavka alebo kódovanie alebo dĺžka nie sú správne, spojenie musí byť ukončené s voliteľnou informáciou o ukončení. Inak musí vrátiť releventnú odpoveď s položkou resultCode nastavenou na príslušnú hodnotu, napríklad \texttt{protocolError(2)}. 

\subsection{MessageID}
MessageID označuje správu jedinečným identifikátorom typu \texttt{integer}. Tento identifikátor je vygenerovaný klientom pri žiadosti a server musí využiť tento identifikátor na všetky odpovede na túto žiadosť. Musí ísť o jedinečnú hodnotu v rámci transakcie.

\subsection{ProtocolOP}
ProtocolOP reprezentuje protokolovú operáciu, ktorá sa má na danom zariadení vykonať. Jednotlivé operácie je možné vidieť v sekcii \ref{message}. Keďže je tento projekt zameraný na vyhľadávanie, v rámci tohoto projektu sú relevantné iba niektoré z nich. Patria medzi ne \texttt{BindRequest, BindResponse, UnbindRequest, SearchRequest a SearchResultDone}. Funkciou väčšiny ostatných operácií je meniť obsah ldap databázy, a teda nie sú podstatné pre riešenie tohoto projektu.

\subsubsection{Operácia Bind}
Bind operácia môže byť prvá operácia, ktorá zahájuje komunikáciu medzi klientom a serverom. Jej hlavnopu funkciou je výmena autentifikačnej informácie. Po prijatí BindRequest žiadosti  server autentifikuje klienta a vráti BindResponse odpoveď so stavom autentifikácie. Od LDAP verzie 3 nie je Bind operácia nevyhnutná a klient môže začať komunikáciu a inou operáciou ako neautentifikovaný. V prípade, že server potrebuje klienta autentifikovať pre vykonanie operácie, môže inú operáciu ako Bind, Unbind alebo rozšírenú žiadosť zakázať pomocou \texttt{resultCode} nastaveného na \texttt{operationError(2)}. Po odpovedi s \texttt{operationError} výsledkom môže klient poslať BindRequest.  Klient môže požiadať o Bind operáciu aj viackrát. V tomto projekte je využitá iba jednoduchá Bind operácia bez autentifikácie \texttt{simple bind}.\\

\begin{figure}[H]
\begin{verbatim}
        BindRequest ::= [APPLICATION 0] SEQUENCE {
                version                 INTEGER (1 .. 127),
                name                    LDAPDN,
                authentication          AuthenticationChoice }
\end{verbatim}
\caption{Formát BindRequest}
\end{figure}

Odpoveďou na BindRequest je BindResponse. BindResponse predstavuje jednoduchú odpoveď, ktorá buď vráti návratový kód ako úspech, alebo nastaví príslušnú chybu. Chybou môže byť napríklad \texttt{operationsError} alebo \texttt{protocolError} pre chybný protokol. Ak klient prijme \texttt{operationError} musí ukončiť spojenie.\\

\begin{figure}[H]
\begin{verbatim}
        BindResponse ::= [APPLICATION 1] SEQUENCE {
             COMPONENTS OF LDAPResult,
             serverSaslCreds    [7] OCTET STRING OPTIONAL }
\end{verbatim}
\caption{Formát BindResponse}
\end{figure}

\subsubsection{Operácia Unbind}
Operácia \texttt{Unbind} slúži na ukončenie komunikácie a nie je párovou operáciou operácie Bind. \texttt{UnbindRequest} nemá žiadnu odpoveď a po prijatí tejto správy môže server ukončiť spojenie.

\subsubsection{Operácia Search}
Na vyhľadávanie záznamov štruktúry LDAP sa využíva operácia \texttt{Search}. Jej základným cieľom je vrátiť záznamy, ktoré spĺňajú vyhľadávané kritériá. 

Požiadavku operácie Search tvorí \texttt{SearchRequest}. Môže čítať atribúty z jedného záznamu, záznamov podriadenej úrovne alebo celého podstromu. Pre potreby tohoto projektu je podporované vyhľadávanie iba v rozsahu jedného záznamu (\texttt{baseObject(0)}). Ďalším vyhľadávacím obmedzením je \texttt{sizeLimit}, ktorý umožňuje určiť maximálny počet vrátených záznamov. Samotné filtrovanie žiadaných nástrojov má na starosti položka \texttt{filter} a požadované atribúty sú definované v položke \texttt{attributes}. \\

\begin{figure}[H]
\begin{verbatim}
        SearchRequest ::= [APPLICATION 3] SEQUENCE {
             baseObject      LDAPDN,
             scope           ENUMERATED {
                  baseObject              (0),
                  singleLevel             (1),
                  wholeSubtree            (2),
                  ...  },
             derefAliases    ENUMERATED {
                  neverDerefAliases       (0),
                  derefInSearching        (1),
                  derefFindingBaseObj     (2),
                  derefAlways             (3) },
             sizeLimit       INTEGER (0 ..  maxInt),
             timeLimit       INTEGER (0 ..  maxInt),
             typesOnly       BOOLEAN,
             filter          Filter,
             attributes      AttributeSelection }
\end{verbatim}
\caption{Formát SearchRequest}
\end{figure}

\noindent Položka \texttt{filter} dokáže vykonať filtrovanie na základe výrazov, ktoré môžu byť do seba niekoľkonásobne zanorené. V tomto projekte sú brané do úvahy iba niektoré filtre. Patria medzi ne \texttt{and(0)}, \texttt{or(1)}, \texttt{not(2)}, \texttt{equalityMatch(3)}, \texttt{substrings(4)} a \texttt{present(7)}. Filtre \texttt{and}, \texttt{or} a \texttt{not} reprezentujú logické operácie konjunkcie, disjunkcie a negácie. Filtre \texttt{equalityMatch} a \texttt{substrings} tvoria výrazy, ktoré musia byť platné pre daný záznam. Prvý spomínaný vyhľadáva iba úplnú zhodu hodnoty atribútu. Filter \texttt{substrings} hľadá na základe regulárneho výrazu podreťazcu.\\ 

\begin{figure}[H]
\begin{verbatim}
        Filter ::= CHOICE {
             and             [0] SET SIZE (1..MAX) OF filter Filter,
             or              [1] SET SIZE (1..MAX) OF filter Filter,
             not             [2] Filter,
             equalityMatch   [3] AttributeValueAssertion,
             substrings      [4] SubstringFilter,
             greaterOrEqual  [5] AttributeValueAssertion,
             lessOrEqual     [6] AttributeValueAssertion,
             present         [7] AttributeDescription,
             approxMatch     [8] AttributeValueAssertion,
             extensibleMatch [9] MatchingRuleAssertion,
             ...  }
\end{verbatim}
\caption{Formát Filter}
\end{figure}

\noindent Regulárny výraz pre filter \texttt{substrings} je uložený v samostatnej položke. Tento výraz je založený na pozícii reťazca alfanumerických znakov. Môžu tvoriť začiatok hodnoty atribútu (\texttt{initial}), vnútro (\texttt{any}) alebo koniec (\texttt{final}).\\

\noindent Príklad:\\

\noindent Regulárny výraz \texttt{str*i*ng} sa pretvorí na postupnosť filtrov \texttt{initial[str]}, \texttt{any[i]} a \texttt{final[ng]}.\\

\begin{figure}[H]
\begin{verbatim}
        SubstringFilter ::= SEQUENCE {
             type           AttributeDescription,
             substrings     SEQUENCE SIZE (1..MAX) OF substring CHOICE {
                  initial [0] AssertionValue,  -- can occur at most once
                  any     [1] AssertionValue,
                  final   [2] AssertionValue } -- can occur at most once
             }
\end{verbatim}
\caption{Formát SubstringFilter}
\end{figure}

\noindent Odpoveďou na požiadavok \texttt{SearchRequest} je \texttt{SearchResponse}. Pomocou tejto odpovede môže server vrátiť nula alebo viac záznamov správami typu \texttt{SearchResultEntry}. Každá takáto správa obsahuje jedinečný názov (DN) vyhľadaného objektu a zoznam vyhľadaných atribútov, ktoré sú definované v \texttt{SearchRequest}.\\

\begin{figure}[H]
\begin{verbatim}
        SearchResultEntry ::= [APPLICATION 4] SEQUENCE {
             objectName      LDAPDN,
             attributes      PartialAttributeList }
             
        PartialAttributeList ::= SEQUENCE OF
             partialAttribute PartialAttribute
\end{verbatim}
\caption{Formát SearchResultEntry}
\end{figure}

\noindent Každá odpoveď \texttt{SearchResponse} musí byť ukončená správou typu \texttt{SearchResultDone}. Táto správa oznamuje, že vyhľadávanie bolo ukončené. Súčasťou tejto správy je položka \texttt{LDAPResult} oznamujúca prípadné chyby v položke \texttt{resultCode}, ako napríklad \texttt{timeLimitExceeded} alebo \texttt{sizeLimitExceeded} pri prekročení limitov. \\

\begin{figure}[H]
\begin{verbatim}
        SearchResultDone ::= [APPLICATION 5] LDAPResult
\end{verbatim}
\caption{Formát SearchResultDone}
\end{figure}

\subsection{Správa výsledku}
Konštrukcia \texttt{LDAPResult} sa využíva na oboznámenie klienta o úspešnosti operácie. V prípade chyby vracia kód, ktorý korešponduje s istým typom chyby v položke \texttt{resultCode}.\\

\begin{figure}[H]
\begin{verbatim}
       LDAPResult ::= SEQUENCE {
             resultCode         ENUMERATED {
                  success                      (0),
                  operationsError              (1),
                  protocolError                (2),
                  timeLimitExceeded            (3),
                  sizeLimitExceeded            (4),
                  ...  },
             matchedDN          LDAPDN,
             diagnosticMessage  LDAPString,
             referral           [3] Referral OPTIONAL }
\end{verbatim}
\caption{Formát LDAPResult}
\end{figure}

\section{Implementácia}
Program serveru je napísaný v programovacom jazyku C verzie 11. Server využíva BSD schránky a je naprogramovaný ako konkurentný server, ktorý pre každého klienta vytvorí vlastný proces. O nastavenie schránky sa stará funkcia \verb|setup_socket|. Zdrojový kód serveru je dostupný v súbore \texttt{server.cpp}. Implementácia samotného LDAP protokolu je definovaná na niekoľkých vrstvách. Základnou triedou, ktorá implemtentuje komunikáciu, je \texttt{LDAP}. Vstupné dáta sú ukladané vo forme poľa typu \verb|uint8_t| a odpoveď sa ukladá do vektoru toho istého typu.  

\subsection{LDAP}
Trieda \texttt{LDAP} je navrhnutá na ovládanie komunikácie LDAP protokolu. Zdrojový kód sa nachádza v súbore \texttt{ldap.cpp}. Na ovládanie komunikácie slúži verejná metóda \texttt{communicate}. Táto metóda spracováva prichádzajúce správy až kým nenastane chyba alebo nebola prijatá správa \texttt{unbind}.

O jednotlivé časti komunikácie sa starajú metódy \texttt{receive} (prijme), \texttt{process} (spracuje) a \texttt{respond} (odpovie). V prípade ukončenia komunikácie sa zavolá funkcia \texttt{respondError}, ktorá v prípade chyby v \texttt{BindRequest} alebo \texttt{SearchRequest} vráti príslušnú správu \texttt{LDAPResult} , ktorá v položke \texttt{resultCode} obsahuje hodnotu \texttt{protocolError(0)}. Ak nebola zistená príslušná operácia, neposiela sa žiadna správa a ukončuje sa spojenie.

Metóda \texttt{receive} spracuje prvé oktety správy a vytvorí objekt triedy LDAPMessage pre nasledovné spracovanie. Prevedenie potrebných operácií a vytvorenie odpovede má na starosti metóda \texttt{process}. Vytvorené správy sú následne odoslané pomocou \texttt{respond} metódy. 

Každý objekt spojenia sa skladá z dvoch operácií, požiadavku \texttt{request} a vektor odpovedí \texttt{response}. Toto je zapríčinené tým, že každá požiadavka má odpoveď s rovnakým messageID.  

\subsection{LDAPMessage}
Spracovanie a vytvorenie nových LDAPMessage obálok vykonáva trieda \texttt{LDAPMessage}. Na spracovanie prijatých dát využíva metódu \texttt{parseMessage} s pomocnou statickou metódou \texttt{parseInt} na spracovanie a prevod celočíselnej hodnoty. Pre spracovanie samotnej operácie sa vytvorí objekt triedy \texttt{LDAPProtocolOp}. 

Trieda LDAPMessage má dva konštruktory, jeden sa stará o uloženie dát prijatej správy, druhý vytvorí LDAPMessage obálku pre objekt triedy \texttt{LDAPProtocolOp} (pre odpoveď). S vytváraním obálky pre správu odpovede je spojená aj pomocná metóda \texttt{appendLengthToMessage}, ktorá pridá do výsledného vektoru oktetov naformátovanú dĺžku dát. Na pridávanie oktetov do finálneho vektoru \texttt{messVector} sa využíva metóda \texttt{appendVector}. 

\subsection{LDAPProtocoOP}
Spracovanie dát na najnižšej úrovni, vyhodnotenie požiadavkov operácie a vytvorenie odpovede prebieha pomocou triedy \texttt{LDAPProtocoOP}. Základ spracovania prichádzajúceho požiadavku tvorí stavový automat definovaný na stránke \texttt{Ldap ASN.1 Codec}\cite{ANS1}. Spracovanie začína metódou \texttt{parseRequest}, ktorá detekuje typ požiadavky. Následne sa ak ide o požiadavku \texttt{BindRequest} alebo \texttt{SearchRequest}, spustí sa buď metóda \texttt{parseBindRequest} alebo \texttt{parseSearchRequest}. V týchto metódach sa prechádza prijatá správa, ktorá je prijatá v poli a prechádzanie tohoto poľa je definované v preťaženej metóde \texttt{incrementIndex}. Táto metóda v prípade prekročenia veľkosti poľa vyhodí vlastnú výnimku \texttt{indexException}.

Na spracovanie jednotlivých položiek typu BER boli vytvorené metódy \texttt{parseString} a \texttt{createString} na spracovanie reťazcov. Na spracovanie veľkosti sa využíva \texttt{parseSizeBER} a na spracovanie celého čísla \texttt{parseIntBER}. Okrem toho ešte existuje niekoľko pomocných metód.

\subsubsection{Spracovanie požiadavkov}
Metóda \texttt{parseBindRequest} iba skontroluje správnosť požiadavky BindRequest, keďže autentifikácia nie je potrebná. Požiadavka SearchRequest sa musím spracovať v metóde \texttt{parseSearchRequest}, aby mohla byť následne prevedená. V tejto metóde sa cez stavový automat postupne prejdú všetky položky požiadavky SearchRequest. Nepotrebné položky sú preskočené a skontroluje sa či ide o \texttt{scope} typu \texttt{baseObject(0)}. Ďalej sa uloží veľkosť \texttt{sizeLimit} a v samostatných metódach spracujú položky \texttt{filter} a \texttt{attributes}. 

Filter sa spracuje v metóde \texttt{parseFilter}, kde sa postupne prejdú všetky filtre prijaté v požiadavke a uložia v štruktúre \texttt{reqStruct}. Táto štruktúra obsahuje veľkosť \texttt{sizeLimit} a pole štruktúr typu \texttt{fItem}. Štruktúra \texttt{fItem} reprezentuje jednotlivé filtre a pamätá si ich operandy, typ a dĺžku. Táto metóda využíva rekurziu na spracovanie všetkých položiek filtru. Na spracovanie podreťazca sa využíva metóda \texttt{parseSubString}, ktorá prevedie túto položku filtra na regulárny výraz.

O spracovanie atribútov \texttt{attributes} sa stará metóda \texttt{parseAttributes}, ktorá tak ako \texttt{parseFilter} pracuje na základe rekurzívneho volania.

\subsubsection{Vykonanie požiadavkov}
Následne sa spracovaná správa vyhodnotí v metóde \texttt{processProtocolOp}, ktorá vráti vektor výsledných operácií. Tieto položky však musia byť ešte zabalené do \texttt{LDAPMessage} obálky. Na požiadavku BindRequest vytvorí jednoduchá odpoveď, keďže nie je potrebné autentifikovať.

Požiadavka SearchRequest však požaduje vyhodnotenie požiadavku a vytvorenie odpovedí typu \texttt{SearchResultEntry} pre každý nájdený záznam a \texttt{SearchResultDone} na ukončenie spracovania požiadavku. Spracovanie súboru databázy na vektor záznamov má na starosti metóda \texttt{parseFile}. Výsledkom tohoto spracovania je sú dva vektory. Jeden obsahuje čísla jednotlivých riadkov pre jednoduchú indexáciu a druhý obsahuje samotné záznamy. Záznamy sú reprezentované vektorom atribútov.

V ďalšom kroku sa vykonanie filtrovanie záznamov na základe požadovaných filtrov v metóde \texttt{parseFilter}. Filtrovanie rovnosti reťazca a podreťazca sa vykoná pomocou štandardnej metódy \texttt{compare} na filtrovanom atribúte a hodnote filtru. Logické operácie sú vykonané pomocou množinových operácií \verb|set_union|, \verb|set_intersection| a \verb|set_difference|.

V poslednom kroku sa na základe nájdených záznamov vytvoria príslušné správy. Vytváraní sa dáva pozor na počet nájdených záznamov, ktorý musí byť menší ako \texttt{sizeLimit}. V prípade prekročenia tejto hodnoty sa vráti iba počet správ definovaný v \texttt{sizeLimit} a správa \texttt{SearchResultDone} bude obsahovať chybový kód \texttt{sizeLimitExceeded} (4). Ako položka DN správy \texttt{SearchResultEntry} sa nastaví uid záznamu a vrátia sa atribúty definované v položke \texttt{attributes} alebo všetky ak neboli žiadne špecifikované. Ak zadaný atribút nebol nájdený, ignoruje sa.


\section{Testovanie}
Prvotné testovanie prebiehalo na osobnom počítači, OS: Arch Linux, gcc v. 7.2. Server bol spustený na lokálnom počítači a následne boli posielané správy pomocou nástroja \texttt{ldapsearch} na tom istom zariadení. Bola využitá množina rôznych výrazov filtru, atribútov a rôznych prepínačov. Správnosť obsahu samotných správ bolo dosiahnuté pomocou nástroja \texttt{Wireshark} s filtrom \texttt{ldap}. Tento nástroj dokáže detekovať ak ide o chybné pakety.

Následne bol tento projekt otestovaný na serveri \texttt{Merlin}, keďže ide o referenčný server. Toto testovanie fungovalo na základe posielania požiadavkov cez \texttt{ldapsearch} zo serveru \texttt{Eva}.

Okrem toho bolo využitý aj jednoduchý TCP klient, ktorý posielal chybné správy na port LDAP serveru a bolo testované, ako na ne zareaguje.

\section{Použitie}
Spustenie serveru:\\

\begin{verbatim}
./myldap {-p <port>} -f <subor>
\end{verbatim}

\noindent Význam parametrov a ich hodnôt:
\begin{verbatim}
    -p <port>: Umožňuje špecifikovať konkrétny port, na ktorom začne 
               server poslúchať požiadavky klientov. Predvolená hodnota
               čísla portu je 389. Voliteľný prepínač.
    -f <soubor>: Cesta k textovému súbor v formáte CSV. Povinný prepínač.
\end{verbatim}

\section{Rozšírenia}
\begin{itemize}
	\item Server dokáže spracovať národné znaky vo formáte UTF-8.
	\item Server vráti vyžiadané atribúty. Príklad:
	\begin{itemize}
		\item \verb|ldapsearch -h hostname -p port -x -s base "(filter)" uid|
		\item Tento príklad vráti iba atribút uid nájdených objektov.
	\end{itemize}
	\item Server dokáže spracovať filter PRESENT (7).
\end{itemize}



\newpage
     
\begin{thebibliography}{9}
\bibitem{rfc2251}
M. Wahl,T. Howes,S. Kille a i. December 1997. Request for Comments: 2251. URL: \texttt{https://www.ietf.org/rfc/rfc2251.txt}
\bibitem{rfc4511}
J. Sermersheim, Ed. Jún 2006. Request for Comments: 4511. URL: \texttt{https://www.ietf.org/rfc/rfc2251.txt}
\bibitem{ANS1}
Emmanuel Lécharny. Október 2006. Ldap ASN.1 Codec. URL: \\\texttt{https://cwiki.apache.org/confluence/display/DIRxSRVx10/Ldap+ASN.1+Codec}

\end{thebibliography}
\end{document}
