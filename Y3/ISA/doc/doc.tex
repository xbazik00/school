\documentclass[11pt,a4paper]{article}
\usepackage[slovak]{babel}
\usepackage[utf8]{inputenc}
\usepackage{amsmath,amssymb}
\usepackage{graphicx}

\title{LDAP}
\author{Martin Bažík}

\sloppy

\begin{document}
\begin{titlepage}
\maketitle
\end{titlepage}
\tableofcontents
\newpage
\section{Úvod}
%https://www.ietf.org/rfc/rfc2251.txt
LDAP je protokol, ktorý poskytuje prístup ku adresáru. Protokol je zacielený na správcovské aplikácie a prehľadávacie aplikácie, ktoré poskytujú vykonávať operácie nad adresárom. Ide o atomické operácie, ktoré sú vykonávané v LDAP správovej vrstve (message layer). Po ukončení spojenia sú nespracované operácie na správovej vrstve zahodené alebo spracované bez odpovede. Synchronizácia akcií medzi klientom a serverom nie sú potrebné.

\textbf{Kľúčové aspekty:}
\begin{itemize}
	\item Opísaný ASN.1 notáciou.
	\item Prenášaný priamo nad TCP alebo iným transportným protokolom
	\item Všetky dátové objekty sú kódované ako bežný reťazec
	\item SASL môže byť použité na zabezpečenie
	\item Hodnoty a názvy musia využívať znakovú sadu ISO 10646
	\item Schéma je zverejnená v adresári pre klienta
\end{itemize}

\subsection{Obálka správy}
Všetky protokolové operácie sú zabalené do obálky LDAPMessage.

\begin{verbatim}
        LDAPMessage ::= SEQUENCE {
             messageID       MessageID,
             protocolOp      CHOICE {
                  bindRequest           BindRequest,
                  bindResponse          BindResponse,
                  unbindRequest         UnbindRequest,
                  searchRequest         SearchRequest,
                  searchResEntry        SearchResultEntry,
                  searchResDone         SearchResultDone,
                  searchResRef          SearchResultReference,
                  modifyRequest         ModifyRequest,
                  modifyResponse        ModifyResponse,
                  addRequest            AddRequest,
                  addResponse           AddResponse,
                  delRequest            DelRequest,
                  delResponse           DelResponse,
                  modDNRequest          ModifyDNRequest,
                  modDNResponse         ModifyDNResponse,
                  compareRequest        CompareRequest,
                  compareResponse       CompareResponse,
                  abandonRequest        AbandonRequest,
                  extendedReq           ExtendedRequest,
                  extendedResp          ExtendedResponse,
                  ...,
                  intermediateResponse  IntermediateResponse },
             controls       [0] Controls OPTIONAL }

        MessageID ::= INTEGER (0 ..  maxInt)

        maxInt INTEGER ::= 2147483647 -- (2^^31 - 1) --
\end{verbatim}

V prípade, že značka LDAPMessage nemôže byť rozoznaná, messageID zanalyzovaná, protocolOp nie je rozpoznaný ako požiadavka alebo kódovanie alebo dĺžka nie sú správne, spojenie musí byť ukončené s voliteľnou informáciou o ukončení. Inak musí vrátiť releventnú odpoveď s resultCode nastaveným na protocolError. MessageID musia byť jedinečné.

\section{Implementácia}
\section{Použitie}
\end{document}
