\documentclass[a4paper,11pt]{report}
\usepackage[slovak]{babel}
\usepackage[utf8]{inputenc}
\usepackage[T1]{fontenc}
\usepackage{latexsym}
\usepackage{hyperref}
\usepackage{float}
\author{Martin Bažík}
\title{Ako získať root na Androide}
\usepackage{graphicx}
\graphicspath{ {images/} }
\frenchspacing
\begin{document}
\section*{Aplikácia pre podporu mentálnych schopností seniorov}
\subsection*{Analýza produktu a cieľov}
Aplikácia pre podporu mentálnych schopností seniorov je aplikácia, ktorej cieľom je pomôcť seniorom udržiavať sa v mentálnej kondícii. Z toho dôvodu bude táto aplikácia zameraná najmä na rôzne hlavolamy. Tieto hlavolamy budú rozdelené na viacero náročností, takže užívatelia sa budú môcť pomaly dostať do tempa až kým ich aplikácia úplne nepohltí. Mentálne schopnosti, ktoré bude táto aplikácia rozvýjať, je možné rozdeliť na viacero oblastí. Medzi tieto schopnosti patrí krátkodobá pamäť, keď si uivatelia budú musieť zapamätať určité obrázky alebo útvary. Ďalej sa bude rozvýjať ich matematické schopnosti na základe jednoduchých matematických výpočtov, ako aj kritické rozhodovanie. 

Od samotného rozhrania si sľubujem najmä prehľadnosť. Keďže cieľovou skupinou sú dôchodcovia je nevyhnutné, aby boli schopný nájsť poskytované funkcie aj bez znalostí počítačov a inej modernej techniky. Samotné grafické prvky musia byť dostatočne veľké a farebne rozlýšené. Seniori majú totiž často problém so zrakom a malé nápisy, či nevýrazné farby môžu by pre nich nečitateľné. Okrem zraku majú seniori často problém aj s jemnou motorikou. Z toho dôvodu je potrebné, aby boli všetky tlačidlá dostatočne veľké.

Cieľovou platformou je webový prehliadač. Táto platforma bola zvolená z toho dôvodu, že nevyžaduje žiadnu dodatočnú inštaláciu a je dostupný prakticky na každom zariadení. Toto je výhoda pre seniorov, keďže sa nemusia trápiť rôzny problémami, ktoré prináša moderná technika, ale môžu sa okamžite ponoriť do tejto aplikácie.

Pre zhrnutie, najdôležitešími atribútmi sú jednoduchosť, veľké prvky a dostatočný farebný kontrast.

\subsection*{Návrh rozhrania}
Samotné rozhranie by sa malo skladať z viacerích vrstiev. Prvou by malo byť jednoduché prehľadné menu, ktoré by poskytovalo len tie najpodstatnejšie prvky. V najlepšom prípade by tam stačili dve tlačidlá. Jedno na zvolenie obtiažnosti a druhé zapnutie samotného hlavolamu. Keďže ide o aplikáciu pre seniorov zlovenie hlavolamu by som nastavil na náhodné, takže by užívatelia nemuseli študovať jednotlivé hlavolamy pred tým ako si ho majú schopnosť vyskúšať. Týmto krokom a zminimalizuje dizajn aplikácie ako aj zväčší priestor pre samotné tlačidlá. 

Po zapnutí daného hlavolamu sa na obrazovke okrem samotného hlavolamu zobrazí iba tlačidlo pre návrat späť do menu a prípadné tlačidlo pre vypnutie alebo zapnutie zvuku.

Rozhranie samotných hlavolamov by malo byť jednoduché a využívať tlačidlá. Tlačidlá sú vhodné z toho dôvodu, že nevyžadujú klávesnicu a dokážu využiť vstup buď z myši alebo z dotyku displeja na dotykovom zariadení. Okrem toho by sa pri každom hlavolame malo zobrazovať aj skóre, ktoré dokáže užívateľa motivovať ku neustálemu zlepšovaniu sa.

\subsection*{Grafický návrh rozhrania}
\end{document}